\chapter{Einleitung}

In der vorliegenden Arbeit werden verschiedene Algorithmen zur Indoor-Ortung basierend auf WiFi-Fingerprints analysiert und implementiert. Dafür werden diese Algorithmen ausführlich in Kapitel \ref{algorithmen} beschrieben, in Kapitel \ref{implementierung} implementiert und in den Kapitel \ref{untersuchungen}, \ref{datenaufbereitung} und \ref{erweiterte_untersuchungen} mit der in Kapitel \ref{testanwendung} beschriebenen Testanwendung analysiert und verglichen. Grundlage der Untersuchungen sind Datensätze, welche in Kapitel \ref{datenerhebung} erfasst wurden. Das Ziel dieser Arbeit ist es, die Genauigkeit der verschiedenen Algorithmen zu untersuchen und zu ermitteln unter welchen Bedingungen und mit welchem Algorithmus in den meisten Fällen eine korrekte Ortung möglich ist. 

Die Ergebnisse dieser Arbeit sollen als Grundlage für die automatischen Indoor-Ortung in der HTW Berlin dienen, sodass zum Beispiel Microcontroller wie der \textit{ESP32} autonom den Raum erkennen können, in dem sie sich befinden. Des Weiteren wird die App \textit{BVG Detection} wieder in Betrieb genommen und dazu verwendet WiFi-Fingerprints zu erfassen und die Position des Nutzers zu bestimmen.

Der verwendete Quellcode ist auf der \textit{GitLab}-Instanz der HTW Berlin unter \url{https://gitlab.rz.htw-berlin.de/s0585012/wifi-fingerprint-based-indoor-localization} gespeichert.
