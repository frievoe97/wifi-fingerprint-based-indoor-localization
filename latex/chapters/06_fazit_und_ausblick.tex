\chapter{Fazit}

\section{Diskussion der Ergebnisse}

Die Ergebnisse dieser Arbeit haben gezeigt, dass unter den untersuchten Algorithmen der gewichtete \gls{knn} Algorithmus mit der Sørensen-Distanzfunktion und \( k  = 5 \) die besten Ergebnisse erzielen konnte und dass die Datenaufbereitungsmethoden zu keiner Verbesserung der Ergebnisse geführt haben. Bei der Analyse von benachbarten Räumen in Abhängigkeit von der Anzahl an Messungen und der Verwendung von Messungen außerhalb der Räume ist diese Arbeit zu dem Ergebnis gekommen, dass dort in nur maximal 51,2 \% der Fälle eine korrekte Ortung möglich war. Aus diesem Grund kommt die Arbeit zu dem Schluss, dass die WiFi-Fingerprint-basierte Raumortung zwar als Unterstützung für die Raumbestimmung genutzt werden kann, jedoch nicht als alleiniges Mittel zur Raumortung verwendet werden sollte.

% Der \gls{accesspoint} sendet die \gls{bssid} und die \gls{ssid}, damit Geräte, wie eine \gls{vm}, sich verbinden können. Algorithmen wie \gls{knn} und \gls{svm} können in einer \gls{api} implementiert werden, um die Datenverarbeitung zu optimieren.

% \section{Zusammenfassung der Ergebnisse}

\section{Ausblick auf zukünftige Arbeiten}

In zukünftigen Untersuchungen könnte dementsprechend überprüft werden, ob mit anderen Techniken wie der Indoor-Ortung via Bluetooth oder Ultra-wideband bessere Ergebnisse erzelt werden können und ob diese vielleicht in Kombination mit der WiFi-Fingerprint-basierten Raumortung genutzt werden können. Ein weiterer Ansatz für zukünftige Arbeiten könnte sein, dass die Geräte, die ihren Raum bestimmen miteinander kommunizieren und die Raumbestimmung nicht allein auf den selbst gemessenen Daten, sondern auch in Abhängigkeit von den Messungen anderer Geräte, basiert, indem die Wahrscheinlichkeiten der Vorhersagen miteinander verglichen werden.

% - Untersuchung von weiteren Methoden zur Indoor Ortung und ob diese vielleicht 
